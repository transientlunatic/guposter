\documentclass{guposter}

%%% Preamble

\authors{
  Author One, School of Physics and Astronomy, University of Glasgow, elder.jpg;
  Author Two, School of Physics and Astronomy, University of Glasgow, watt.jpg;
  Author Three, School of Somewhere Else, University of Otherplace, kelvin.jpg
}

\logos{logo.png,logo.png,logo.png}
\title{A Glasgow University Poster}
%\subtitle{Using Gaussian Processes to Motivate Numerical Relativity Simulations}


\begin{document}
\maketitle
%\drawgrid

%%%
%%% Abstract
%%%
 
\begin{abstract}
  \textrm{\textbf{This \LaTeX{} style file is intended to make producing
      posters for conferences easy.} Where possible it attemptes to
    stick roughly to the University of Glasgow's visual branding
    guidelines, but makes some changes to suit the medium.\\[2em]
    This text block is indicative of how the poster abstract would look
    under the default configuration. There is space for some text, and
    then to the right a space where an image or project logo could be
    inserted.  }
\end{abstract}

 \begin{panel}{d0}{3}[university-burgundy]
   {\Large \texttt{GUPoster} is designed around the concept of a ``grid''
     system (and it was more directly inspired by HTML and CSS design
     systems like \texttt{Bootstrap}), but with suitable modifications
     for printing rather than on-screen design. }
 \end{panel}

 \begin{panel}{d4}{3}[red]
   {\Large Simple text blocks can be defined with a position, and with
     a width; on this row there are three equal-width blocks, but that
     is not a requirement.}
 \end{panel}

 \begin{panel}{d8}{3}[university-rust]
   {\Large The colour of the ribbon feature at the side of each text
     box can also be cusomised easily, to allow subtle hinting at
     related concepts across the page.}
 \end{panel}
 
 \begin{fillpanel}{g0}{5}[university-lavendar][black]
   It's also possible to included ``filled'' panels, which have a
   coloured background. These can be quite eye-catching, but it's
   probably not a good idea to use more than one or two, as the
   coloured background can reduce the readability (and thus the
   accessibility) of the poster.
 \end{fillpanel}

  \begin{fillpanel}{h0}{5}[university-rust][white]
   In order to not be too jarring, the background colour of these
   panels is set to 50\% opacity, which should provide a more
   pastel-esque esthetic to the poster.
  \end{fillpanel}
  
  \begin{fillpanel}{i0}{3}[university-lawn][black]
    There are many colours to choose from, but make sure that you
    choose a foreground text colour which can be read above the
    bakground.
  \end{fillpanel}

  \begin{fillpanel}{i3}{2}[university-cobalt][black]
      Other than that, these panels are just a whole lot of fun.
 \end{fillpanel}
 
 \begin{panel}{g7}{4}
   The design also allows for easy inclusion of author photographs,
   and authors from multiple Institutions. Logo placement at the
   bottom of the sheet is also handled automatically.
 \end{panel}

  \begin{panel}{i5}{6}
    Each of the anchor points in the poster can be accessed directly
    from tikz, so it's possible to easily add illustrations and more
    complex graphical elements directly onto the page.
    
 \end{panel}

  \begin{tikzpicture}[remember picture, overlay]
    \draw [ultra thick, university-rust] (k6) -- (k8) -- (j9) -- (p6);
  \end{tikzpicture}
  

 
\end{document}
